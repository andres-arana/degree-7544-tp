\documentclass[a4paper,11pt]{article}

%%%%%%%%%%%%%%%%%%%%%%%%%%%%%%%%%%%%%%%%%%%%%%%%%%%%%%%%%%%%%%%%%%%%%%%%
% Paquetes utilizados
%%%%%%%%%%%%%%%%%%%%%%%%%%%%%%%%%%%%%%%%%%%%%%%%%%%%%%%%%%%%%%%%%%%%%%%%

% Gráficos complejos
\usepackage{graphicx}
\usepackage{caption}
\usepackage{subcaption}
\usepackage{placeins}

% Soporte para el lenguaje español
\usepackage{textcomp}
\usepackage[utf8]{inputenc}
\usepackage[T1]{fontenc}
\DeclareUnicodeCharacter{B0}{\textdegree}
\usepackage[spanish]{babel}

% PDFs embebidos para el apéndice
\usepackage{pdfpages}

% Tablas complejas
\usepackage{multirow}

% Formato de párrafo
\setlength{\parskip}{1ex plus 0.5ex minus 0.2ex}

%%%%%%%%%%%%%%%%%%%%%%%%%%%%%%%%%%%%%%%%%%%%%%%%%%%%%%%%%%%%%%%%%%%%%%%%
% Título
%%%%%%%%%%%%%%%%%%%%%%%%%%%%%%%%%%%%%%%%%%%%%%%%%%%%%%%%%%%%%%%%%%%%%%%%

% Título principal del documento.
\title{\textbf{Ejercicio Práctico 01: Project Charter}}

% Información sobre los autores.
\author{
  \normalsize{\textbf{1er. Cuatrimestre de 2014}} \\
  \normalsize{\textbf{75.44 Adm. y Control de Proyectos Informáticos}} \\
  \normalsize{\textbf{Facultad de Ingeniería, Universidad de Buenos Aires}} \\
  \\
  Andrés Gastón Arana, \textit{P. 86.203} \\
  Pablo Oddo, \textit{P. 78.368} \\
  Camilo Barraza, \textit{Intercambio} \\
  Morales Oviedo Karlol, \textit{P. 87.654} \\
}
\date{}

%%%%%%%%%%%%%%%%%%%%%%%%%%%%%%%%%%%%%%%%%%%%%%%%%%%%%%%%%%%%%%%%%%%%%%%%
% Documento
%%%%%%%%%%%%%%%%%%%%%%%%%%%%%%%%%%%%%%%%%%%%%%%%%%%%%%%%%%%%%%%%%%%%%%%%

\begin{document}

% ----------------------------------------------------------------------
% Top matter
% ----------------------------------------------------------------------
\thispagestyle{empty}
\maketitle

\begin{abstract}

  El presente documento constituye el Charter para el proyecto "Gaviota
  Ballena", estableciendo los objetivos, alcance y participantes del mismo.

\end{abstract}

\clearpage

% ----------------------------------------------------------------------
% Tabla de contenidos
% ----------------------------------------------------------------------
\tableofcontents
\clearpage

% ----------------------------------------------------------------------
% Desarrollo
% ----------------------------------------------------------------------

\section{Propósito}

El proyecto “Gaviota Ballena” tiene como fin erradicar de manera controlada la
proliferación de gaviotas en el área oceánica de la población de Madryn con el
menor impacto ambiental posible de manera que no afecten a la población de
ballenas.

La presencia de las ballenas en el área de Madryn es un incentivo de suma
importancia para el turismo. La proliferación descontrolada de las gaviotas
amenaza con la supervivencia de las mismas, y con este, la continuidad
económica de la región.

\section{Objetivos}

Se definen los siguientes objetivos:

\begin{itemize}

  \item Reducir en un 50\% el porcentaje de ballenas lesionadas a causa de las
    heridas que les ocasionan las gaviotas al finalizar la implementación del
    proyecto.

  \item Reducir la cantidad de gaviotas en la zona costera en un 50\% al
    finalizar la implementación del proyecto.

\end{itemize}

\section{Requisitos}

Los requerimientos de alto nivel del proyecto son los siguientes:

\begin{itemize}

  \item Reubicar los basureros a cielo abierto para disminuir la presencia de
    gaviotas en el área costera.

  \item Implementar la infraestructura para permitir el reciclaje de la basura
    generada por la población, así como los cambios culturales necesarios para
    hacerlo efectivo.

  \item Diseñar y promover políticas para el control de la disposición de
    basura en el área oceánica de Madryn, en particular la generada por los
    botes pesqueros y sus bodegas.

\end{itemize}

\section{Riesgos}

Se observan los siguientes riesgos:

\begin{itemize}

  \item La posibilidad de trasladar los basurales a cielo abierto se encuentra
    condicionado a la determinación de lugares apropiados para los nuevos
    basurales, en donde su impacto ambiental sea mínimo, lejos de las costas.

  \item La resistencia cultural a la implementación de un régimen de reciclado
    puede influir negativamente en el proyecto, disminuyendo ampliamente el
    efecto de control de proliferación buscado.

  \item La aparición de factores ambientales no controlables foráneos a la
    producción de basura que favorezcan la proliferación de las gaviotas.

  \item La presión de grupos económicos desfavorecidos por las medidas de
    administración y control de la basura, en particular los lobbies de
    industrias pesqueras e industrias terrestres de la región, que pueden poner
    en riesgo la implementación del proyecto y la efectividad de las medidas.

\end{itemize}

\section{Supuestos}

Para la confección de este documento se tomaron en cuenta los siguientes
supuestos:

\begin{itemize}

  \item Existen lugares alternativos a donde trasladar los basurales a cielo
    abierto para alejarlos de las costas marítimas.

  \item Se cuenta con la cooperación de las áreas gubernamentales
    correspondientes, como ser las secretarías a cargo de la definición de
    políticas de manejo de basura, los responsables de la planificación de
    infraestructura para su manejo y los encargados de canales de comunicación
    oficiales con el pueblo.

  \item Existe la disponibilidad de fondos para encarar la translación de los
    basurales a cielo abierto y la construcción de plantas de reciclado.

  \item Se tiene la colaboración de las fuerzas armadas y otras fuerzas de
    seguridad encargadas de llevar a cabo el control del cumplimiento de las
    políticas a definir en tierra y espacio marítimo.

\end{itemize}

\section{Cronograma de Hitos}

A partir de la aprobación del proyecto antes de las dos semanas de emitido este
documento, se observaran los siguientes hitos, detallados en cantidad de
semanas después de la aprobación:

\begin{itemize}

  \item \textbf{2da semana}: Inicio del proyecto.

  \item \textbf{4ta semana}: Aprobación de Alcance y Presupuesto. Inicia
    implementación del proyecto.

  \item \textbf{38va semana}: Finalización de implementación del proyecto.
    Inicia finalización y medición de resultados finales.

  \item \textbf{40ma semana}: Finalización del proyecto.

\end{itemize}

\section{Resumen de Presupuesto}

Se observan los siguientes conceptos que influyen directamente en el
presupuesto:

\begin{itemize}

  \item Construcción y puesta en marcha de planta de reciclaje.

  \item Adquisición de terrenos para la reubicación de los basureros a cielo
    abierto.

  \item Personal especializado y profesionales para la implementación del
    proyecto.

  \item Diseño e implementación de campañas publicitarias y de concientización
    del manejo de la basura.

\end{itemize}

\section{Partes Implicadas}

Los siguientes son los actores principales implicados en el proyecto:

\begin{itemize}

  \item Intendente de Madryn.

  \item Gobernador de Chubut.

  \item Prefectura Nacional.

  \item Armada.

  \item Población de Madryn.

  \item Empresas de Construcción de planta de reciclado y acondicionamiento de
    los nuevos basurales a cielo abierto.

\end{itemize}

\end{document}

